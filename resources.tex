\section{Medios hardware}
Los medios hardware que se van a utilizar para el desarrollo del proyecto son
los siguientes:
\begin{itemize}
\item \textbf{Lector \emph{NFC}}.
\item \textbf{2 Ordenadores de sobremesa}. Uno de ellos estará conectado al
  lector \emph{NFC} a través de uno de los puertos \emph{USB}. El otro
  contará con la tecnología \emph{Bluetooth}, ya sea instalada en el propio
  equipo o mediante un dispositivo externo conectado a uno de los puertos
  \emph{USB}. Ambos equipos estarán conectados a la misma red local, que a su
  vez estará conectada a internet.
\item \textbf{\emph{Nokia 6131 NFC}}. Este dispositivo cuenta con las
  tecnologías inalámbricas \emph{Bluetooth} y \emph{NFC}.
\item \textbf{Etiquetas \emph{RFID}}. Las etiquetas son del tipo \emph{MiFare},
  de 1KB de capacidad.
\end{itemize}

\section{Medios software}
Para desarrollar el proyecto se utilizarán las siguientes herramientas
software:
\begin{itemize}
\item \textbf{\emph{Microsoft Visual Studio 2010}}. Este herramienta de
  desarrollo se utilizará para implementar las aplicaciones \emph{WPF} de los
  equipos de sobremesa. Las aplicaciones serán desarrolladas en \emph{C\#}. La 
  versión del \emph{Framework} de \emph{.NET} será la 3.5.
\item \textbf{\emph{Netbeans IDE 7.0.1}}. Este IDE se utilizará para
  desarrollar la suite de \emph{Midlets} del dispositivo móvil
  \emph{Nokia 6131 NFC}. Los \emph{Midlets} son aplicaciones implementadas
  en el lenguaje \emph{JavaME} (\emph{Java Micro Edition}. La aplicación estará
  orientada a dispositivos con configuración \emph{CLDC-1.1} y perfil
  \emph{MIDP-2.0}. Para implementar las comunicaciones \emph{NFC} se hará uso de
  la librería \emph{jsr257}.
\item \textbf{\emph{SDK Nokia 6131 NFC}}. Este kit de desarrollo será utilizado
  para simular la suite de \emph{Midlets} construida, antes de pasarla al
  dispositivo físico real.
\end{itemize}


