\section{Metodología de desarrollo}
Para llevar a cabo el desarrollo de este proyecto se utilizará una adaptación
de la metodología de \emph{prototipado incremental} \cite{bib:PI}. Esta
metodología permitirá una construcción iterativa e incremental de cada una de
las aplicaciones que componen el sistema, controlando la complejidad y los
riesgos; y dejará abierta la posibilidad de realizar nuevas ampliaciones en un
futuro. 

La metodología de prototipado incremental propone la realización una serie de
mini-cascadas, donde las principales fases de desarrollo de una metodología
de cascada son completadas, para una pequeña parte de los sistemas, antes de
proceder a la siguiente iteración. Cada iteración termina con la construcción
de un nuevo prototipo o la ampliación de un prototipo anterior.

En nuestro caso, se establece el diseño y desarrollo de prototipos móviles (en forma de MIDlets) y otras partes del sistema (Gestión y comunicación).

\section{Fases}
Las \emph{figuras \ref{img:Gantt1}} y \emph{\ref{img:Gantt2}} muestran el
tiempo de dedicación previsto para la realización de las diferentes tareas y
actividades que forman parte del proyecto.

\begin{sidewaysfigure}
\imagen{gantt1.png}{21}{Diagrama de Gantt (Diciembre-Febrero)}{img:Gantt1}
\end{sidewaysfigure}

\begin{sidewaysfigure}
\imagen{gantt2.png}{21}{Diagrama de Gantt (Marzo-Junio)}{img:Gantt2}
\end{sidewaysfigure}


%\section{Metodología de desarrollo}
%Para llevar a cabo el desarrollo de este proyecto se utilizará el
%\textbf{Proceso Unificado de Desarrollo} \cite{bib:PUD}. Esta metodología de
%desarrollo permitirá una construcción iterativa e incremental de cada una de
%las aplicaciones que componen el sistema.

%Como decimos, al tener que desarrollar varias aplicaciones, se empezará
%desarrollando una de las aplicaciones. Cuando esté implementada la
%funcionalidad básica de esta aplicación, se empezará a desarrollar otra de
%las aplicaciones y las comunicaciones con la primera aplicación. Y así con
%cada una de las aplicaciones que conforman el sistema.

%Una vez implementada la funcionalidad básica y las comunicaciones de todas
%las aplicaciones, se procederá a terminarlas y probarlas.

%\section{Fases}
%La elaboración del proyecto al completo estará dividida en las siguientes
%etapas:
%\begin{itemize}
%\item \textbf{Estado del Arte}: se realizará un estudio sobre la situación y
%progresión actual de aplicaciones similares al PFC además de diversos sistemas
%e investigaciones que puedan servir como fuentes de inspiración.
%\item \textbf{Desarrollo software}: en esta parte del proyecto se llevará a
%cabo todo el proceso de construcción de la aplicación en sí. Por cada iteración
%se cumplirán las siguientes fases:
%\begin{itemize}
%\item \textbf{Captura de requisitos}. Se obtendrá el modelo de casos de uso a
%partir de los requisitos funcionales.
%\item \textbf{Análisis}. Se analizarán los requisitos capturados en fases
%anteriores y se estructurarán de tal modo que se obtenga una descripción para
%permitir una mejor comprensión de los mismos.
%\item \textbf{Diseño}. En esta fase se modelará el sistema y su arquitectura.
%Será indispensable en su entrada el modelo de análisis de la fase anterior.
%\item \textbf{Implementación}. Se parte del resultado del diseño para
%implementar el sistema en términos de componentes como el código fuente,
%ejecutables, binarios, etc.
%\item \textbf{Pruebas}. Para cada iteración se planificarán las pruebas
%necesarias para poder implementarlas creando los casos de prueba y por último,
%ejecutarlas.
%\end{itemize}
%\item \textbf{Evaluación y documentación}: se someterá al software a una
%evaluación con usuarios finales para comprobar el comportamiento de la
%aplicación, además de la reacción y respuesta por parte de los mismos. Por
%último se documentará la parte técnica del proyecto.
%\end{itemize}

