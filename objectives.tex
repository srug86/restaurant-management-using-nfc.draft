\section{Introducción}
En los últimos años, el uso de dispositivos móviles se ha convertido
en un acto totalmente cotidiano; y esto es un hecho en gran parte del planeta.
Y es que, desde que aparecieron los primeros modelos en los años 80, el número
de usuarios se ha incrementado prácticamente a la misma velocidad que lo han
hecho los servicios que estos ofrecen.

Uno de los últimos avances en ser incorporados ha sido la tecnología
\emph{\textbf{NFC}} (\emph{Near Field Communication}). \emph{NFC} es una
tecnología de comunicación inalámbrica, de corto alcance y alta frecuencia que
permite el intercambio de datos entre dispositivos. Es una simple extensión
del estándar \emph{ISO 14443} (\emph{RFID}) \cite{bib:RFID}.

En este PFC se pretende hacer uso de esta tecnología para complementar,
facilitar y enriquecer las principales tareas que se llevan a cabo en un
restaurante en la interacción con sus clientes, entre ellas: asignación de
mesas, realización de pedidos y el pago de estos.

\section{Objetivos del proyecto}
\subsection{Hipótesis general}
Se pueden mejorar y complementar los sistemas actuales de gestión de mesas, 
menús y atención de clientes en restaurantes, mediante el uso de nuevas 
tecnologías integradas en dispositivos universales

\subsection{Objetivo general}
Este PFC tiene como objetivo diseñar e implementar un sistema que sea capaz de
automatizar, complementar y enriquecer algunas de las tareas que se llevan a 
cabo en un restaurante, con objeto de aumentar su productividad. Para
conseguirlo, se hará uso del potencial que poseen los dispositivos móviles y
tecnologías inalámbricas como \emph{NFC} o \emph{Bluetooth}. Esto permitirá
obtener nuevas formas de interacción entre los clientes y el restaurante, pero
sin poner en riesgo para ello la sencillez o la naturalidad del proceso.

El sistema a desarrollar no busca sustituir a los sistemas existentes en la
actualidad, sino que tratará de complementarlos.


En la \emph{figura \ref{img:MAP}} puede verse un esquema general del flujo del
sistema.

\imagen{map.png}{13}{Esquema general del sistema}{img:MAP}

Pasos:
\begin{enumerate}
\item El cliente ``toca'' con su dispositivo móvil el lector \emph{NFC} del
  primer PC y queda registrado en este con un identificador único. El PC
  muestra las mesas que están libres (y su capacidad) para que el cliente elija
  entre una de ellas.
  Se tendrá a un \emph{maître} a cargo de este primer PC. Este se encargará de
  ayudar a los usuarios que no estén todavía familiarizados con el sistema y
  registrará, de forma manual, a los usuarios que no dispongan de un
  dispositivo con \emph{NFC} y/o \emph{Bluetooth}.
\item Ya en la mesa, el cliente ``toca'' con su dispositivo móvil las etiquetas
  \emph{RFID} de la carta, para generar una lista con los productos que este
  pretende consumir. Cuando ha terminado de elegir el pedido, lo manda vía
  \emph{Bluetooth} al segundo PC.
  Los usuarios que no hayan registrado su dispositivo previamente,
  realizarán los pedidos de forma tradicional, es decir, avisando a alguno de
  los camareros.
\item El segundo PC se encarga de recibir los pedidos de los clientes que
  fueron registrados por el primer PC y de acumular el importe total de cada
  uno de ellos.
  Los pedidos atendidos por los camareros son registrados manualmente a través
  de la interfaz gráfica de la aplicación del segundo PC, asegurando la 
  integración y convivencia de sistemas.
\item Cuando el cliente ha terminado, tiene la posibilidad de realizar el pago
  mediante su dispositivo móvil. Para ello, se dirigirá de nuevo al lector 
  \emph{NFC} del primer PC.
  Los usuarios que no puedan o no quieran hacerlo de esta manera, podrán
  hacerlo siguiendo el método tradicional.
\end{enumerate}

\subsection{Objetivos específicos}
Para la realización del PFC se han marcado los siguientes objetivos:
\subsubsection{Análisis}
\begin{enumerate}
\item Realizar un estudio de campo sobre el funcionamiento actual de los
  restaurantes. Cómo se organizan, cómo operan y qué medios utilizan para la
  realización de tareas.
\item Realizar una revisión sistemática de las aplicaciones existentes en el
  mercado que compartan las mismas premisas que nuestro sistema, para ver qué
  ofrecen y para ver cuáles son las ventajas y desventajas que implica elegir
  entre uno u otro sistema.
\item Identificar los requisitos que deben cumplir cada uno de los elementos
  que forman parte de nuestro sistema. Así como también los requisitos que
  deben satisfacer todos ellos funcionando como un conjunto.
\item Estudiar la forma de integrar este sistema con algún sistema que el
  restaurante ya tuviese implantado.
\item Estudiar la forma de introducir este sistema para clientes que no
  dispongan de un dispositivo adecuado.
\item Estudiar una forma de incentivar el uso del sistema de pago mediante
  el dispositivo móvil (descuentos, programas de fidelización, etc.).
\end{enumerate}

\subsubsection{Adquisición de conocimientos}
\begin{enumerate}
\item Adquirir los conocimientos necesarios para implementar interfaces de
  usuario utilizando la tecnología \emph{WPF} (\emph{Windows Presentation
  Foundation}).
\item Conocer las peculiaridades del lenguaje \emph{\textbf{Java Micro
  Edition}} para la implementación de \emph{MIDLets} para dispositivos móviles
  con soporte Java.
\item Conocer las principales clases y métodos que ofrece la librería
  \emph{jsr257} para operar con etiquetas RFID y dispositivos NFC del entorno.
\item Conocer y usar las librerías implementadas como parte del estándar para
  \emph{JavaME} que permiten las comunicaciones vía \emph{Bluetooth} entre 
  dispositivos (JSR-82).  
\end{enumerate}

\subsubsection{Diseño e Implementación}
\begin{enumerate}
\item Diseñar e implementar una aplicación de escritorio táctil que permita
  configurar y actualizar el estado de las mesas del restaurante. Esto
  incluirá el diseño e implementación de las comunicaciones entre el lector
  de \emph{NFC} y la aplicación.
\item Implementar la forma de realizar el pago mediante el dispositivo móvil,
  utilizando para ello una interacción vía \emph{NFC} con el lector del equipo
  de la entrada.
\item Diseñar e implementar una aplicación de escritorio táctil que permita
  registrar los pedidos de los clientes (de forma manual o automática - vía
  \emph{Bluetooth}), actualizar el estado de las mesas (con la información de
  los pedidos) y generar el importe total de cada cliente.
\item Diseñar e implementar una suite de \emph{MIDLets} que genere una lista
  de productos en el móvil a partir del contacto del dispositivo con las
  etiquetas \emph{RFID} que forman parte del menú. La aplicación también
  permitirá enviar al equipo de la barra el pedido vía \emph{Bluetooth}.
\item Diseñar la las estructura de los datos necesaria para representar la
  información de los platos del menú y de otras opciones (pedir la cuenta, 
  mostrar recomendaciones, etc.) dentro de las etiquetas \emph{RFID}.
\item Desarrollar un módulo en el propio sistema para la creación de
  recomendaciones atendiendo a las preferencias del cliente y a visitas
  anteriores al restaurante.
\end{enumerate}

\subsubsection{Pruebas}
\begin{enumerate}
\item Realizar pruebas unitarias y de integración, para cerciorarnos de que
  todos los elementos del sistema funcionan.
\item Realizar pruebas de sistema, para comprobar que los elementos que
  componen el sistema pueden comunicarse.
\item Realizar pruebas funcionales, para comprobar que las respuestas del
  sistema son las esperadas, dadas una serie de entradas.
\item Realizar pruebas de operación, con el sistema en ejecución y totalmente
  operativo.
\end{enumerate}

\section{Motivación}
El desarrollo de este PFC intenta demostrar cómo pueden aplicarse las
distintas tecnologías que poseen los dispositivos móviles actuales, para
desarrollar nuevas formas de interacción, que permitan mejorar procesos
que se realizan habitualmente (como el caso de realizar pedidos o pagos en un
restaurante) y además, que lo hagan de una forma sencilla, cómoda e intuitiva.

Hoy en día, gran parte de las empresas de restauración cuentan con aplicaciones
que les ayudan a automatizar la anotación de pedidos y la generación de
facturas (e incluso conocer el estado de las mesas \cite{bib:DHC}). Pero estos
sistemas simplemente acomodan el uso del clásico lápiz y papel y la máquina
registradora. Este sistema va un paso más allá y trata de mejorar visiblemente
la productividad del proceso así como un enriquecimiento del entorno, además
de proponer una forma innovadora de realizar pedidos.

Actualmente, hay algunas implementaciones que buscan automatizar la forma
de realizar pedidos, como \emph{vMenu} (de la empresa \emph{vloo}), que
utiliza un \emph{iPad}, tablet o pantalla, como carta para realizar pedidos 
\cite{bib:vMenu}. La solución que se propone en el presente PFC, parte de la
misma idea pero teniendo en cuenta una mínima interacción del usuario con el 
dispositivo para la obtención de servicios (p.e. petición de pedidos), una 
mayor versatilidad (incluyendo la realización de pagos\cite{bib:Pay} y 
generación de recomendaciones), y un menor coste (pues no hace falta disponer 
de tantos dispositivos como mesas haya).


